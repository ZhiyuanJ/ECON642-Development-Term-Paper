\section{Conclusion and Limitation}
\label{Conclusion}
\subsection{Conclusion}

With this study, we cast doubt on some previous research about female and their influence on the country-level economic development.
We found no obvious evidence to suggest female politicians, especially the lawmakers of the legislature, has a positive influence on a country's budget allocation of healthcare.
This result is consistent across most development levels, democracy levels, and healthcare systems. However, we did find a relatively weak but strongly significant positive effect of female lawmakers in Lower-middle income countries'\footnote{countries with GNI per capita between 750 U.S dollars to 4000 U.S. dollars} healthcare expenditure.
We also analyze the results for countries with different democratic levels.
The evidence proves that only in countries where the democratic system is under development, the influence of female lawmakers is significant. 

In conclusion, the influence of female lawmakers is limited in most countries.
Even the past research suggests a policy preference, the females can rarely determine the share of healthcare expenditure of a country.
However, in countries where still under development both economically and politically, women can still play a crucial role in the issue of healthcare.

When discussing the influence of the democracy level and the leader's gender, we created a difference-in-difference (DID) liked method to evaluate the impact. 
The DID method relies on several assumptions and a specific experimental-like setting which may not meet the scenario this research faces. 
Therefore, the results from the corresponding section may need a more careful interpretation and consideration. 

\subsection{Limitation and Potential Extension}
As the study continues, many limitations that restrict this paper's development appear. 

%Sample size (both time horizon and the country candidates)
The first problem is the sample size.
After balancing the quality and quantity of the data, the timeframe of the data set had been restricted by 20 years.
However, the influence of women's empowerment, especially from a high-level perspective, might need a longer time period to be reflected in the reality, hence a data set with more observations will be beneficial for this project.

%Control variables (various control, democratic measurement, government measurement)
The selection of the control variables could be improved in different ways.
Various controls should be considered for testing the robustness of the result.
For instance, the female labor participation rate as control of the "better system of better gender" argument can be replaced by the female completion rate of secondary education or other variables.
The measurement of the democracy level in this study is inevitably biased since the evaluation of political freedom is subjective.
Hence, different variables could be considered for the democracy variable.
In the case study of the budget maker and budget approver, only the gender of the country leader had been considered.
To have a better understanding of the issue, more variables that sketch the administrative branch shall be employed (for example, the method used by \citeNP{Mavisakalyan2014}).

%Selection of measurement
By studying other papers, we also found the selection of measurement might significantly influence the results of the paper.
For instance, \citeA{Chen2010} used the number of available hospital beds as a measurement of healthcare quality when studying the influence of introducing a gender quota to the national parliament.
That says the measurement of this study might not be a fair measurement of the country's real health care expenditure, and hence could be adjusted by using other measurements. 

%endogenous (democratic, richness, and political participation)

Endogeneity is hard to avoid in this study.
Especially considering the dynamic at the democratic level, the wealth level of a country, and the opportunity for females to participate in high-level politics.
Although some approaches had been used to address the endogenous problem like the inclusion of control variables, other methods are also worth consideration.
For example, using Instrumental Variables to replace some endogenous variables may provide a better regression result and a better understanding of the problem.

%omitted variables (especially for the low-income group)
In the section \ref{Grouping}, we mentioned that the Low-Income countries group might suffer omitted variables biased, and they should be given a different set of explanatory variables for analysis.
That says the current result for that group of countries might be biased and unreliable, and hence need a careful interpretation.

With regard to the above limitations, we also provide some suggestions about the future extension of this project.

%Expand the dataset (both countries and year)
The first is to expand the dataset.
As discussed in the limitation, influence from the changing political landscape may need a longer timeframe to examine.
Therefore, future researchers could consider either including earlier data points or collecting the latest data to re-do the model. 

% Other sectors (education social welfare)
As suggested by \citeA{Gerdtham2000} and \citeA{Hitiris1992}, the health care expenditure is dominated by several factors, and hence only left limited space for other factors.
It does not mean that female lawmaker is insignificant to the country's budget allocation.
By focusing on other issues like the education expenditure or the social welfare project, we could obtain a more holistic understanding of the relationship between women's empowerment and the country's development. 

%Endogenous problem and find IV

%Better description of the administrative branch

% Separate study of the poor country
We also suggested isolating the low-income countries from other countries.
Regression results suggest different behaviors between low-income countries and countries from other income levels.
The current model may fail to capture the idiosyncratic characteristics of the low-income countries, and we suggest re-formulized models for this specific group. 




