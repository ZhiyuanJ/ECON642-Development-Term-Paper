\section{Introduction}
In 2015, United Nations (UN) presented a new framework aiming to improve the sustainable development of the world, naming it as "Sustainable Development Goals" (SDGs).
This framework includes 17 bulletins, and the fifth goal is "achieve gender equality and empower all women and girls".
%TODO add some transmission to connect the beginning and the rest 
This high priority of gender equality indicates the importance of this issue with regard to the general development of human welfare.

In fact, the relationship between economic development and the empowerment of women had been long discussed by various scholars.
\citeA{Duflo2012} documented a series of papers researching the relationship between women's empowerment and economic development.
In general, the research agrees on a positive relationship between women's empowerment and economic development, and the relationship goes in both ways.
Narrowing the gender difference relies on the development of the social-economic conditions of a country, and such improvement will benefit the country's long-term economic development.
Especially in some specific filed, like nutrition and education, the rising status of females will bring a great boost to the country's economy.
For example, \citeauthor{Duflo2012} admitted that although with limitations, women's empowerment can lead to improvements in children's health.

One potential mechanism that enables such improvement is budget allocation.
Women can influence the item being purchased by the household, and therefore influence the welfare.
This conclusion, from the micro-level, had been supported by numerous researchers.
\citeA{Hoddinott1995} studied the data from Cote d'Ivoire and found that women tend to spend household income on family-friendly items such as nutrition rather than alcohol or cigarette.
And such spending increase has a positive influence on children's nutrition condition.
Similarly, \citeA{Quisumbing2003} argues that in some countries the increase in a household's female members' assets will lead to an increase in expenditure on children's education. 
The evidence above arguably proved that at least at the household level, females tend to be better budget makers than males on the issues of health and education.

The study about women's influence does not stop at the household level, but the presentation of women in other fields has also been proved to have a positive influence.
In the middle and lower levels of government management, such as state legislation \cite{Irma2011} or village council \cite{Chattopadhyay2004}, women leaders tend to have different preferences for public goods allocation and infrastructure investment plans.
Those preferences include more investment in projects related to nutrition, entry-level education, and policies that guaranteed the equally inherent right of males and females, etc.


With the hierarchy going up, scholars had observed more evidence to support the argument that women's participation in national-level governing might bring benefits. 
\citeA{Dollar2001} provided evidence to show that the involvement of female officers in government can reduce the overall corruption level of the administrative body.
But this result had been questioned by other scholars (such as \citeNP{Sung2003}) since this effect can be caused by a better democratic system that encourages more women's participation and dampens the corruption, rather than the other way around. 
Taking this measurement into account, however, later research still finds the benefits of involving women in country-level decision-making.
Especially in the legislative body where lawmakers are responsible for making laws and designing budgets.
\citeA{Jayasuriya2013} collected data from over 100 countries and concludes that the country with a higher participation rate of women in the law-making process tends to have a higher economic growth rate in general.
As a subjective indicator, \citeA{York2014} presented the result that people tend to have a higher life-satisfaction rate if their national parliament or house of deputies has a higher ratio of female members.  
The influence of female lawmakers also extends to other more specific fields.
For example, \citeA{Salahodjaev2020} showed an "S" shape relationship between a country's deforestation level and the proportion of women members in the legislative body. 
The deforestation will decrease as the number of female lawmakers increases.
Such decreasing will meet some bottlenecks, but the deforestation rate will keep moving down once the ratio breaks a certain threshold. 

With all the research presented above, the relationship between women's participation in the parliament and the national health expenditure has rarely been discussed. 
Current studies more focusing on the general budget allocation plan or the administrative branch.
\citeA{Funk2018} analyzed data from over 5000 Brazil municipalities and found that female mayors and legislative representatives tend to spend more budget on feminine issues, like health care and education, and less on masculine issues such as transportation.
Although the distinction between feminine issues and masculine issues is vague and subjective, this finding is aligned with the similar studies in India \cite{Irma2011}.
To the country level legislation body, \citeA{Chen2010} found same result as \citeauthor{Funk2018}, and \citeauthor{Irma2011}.
By studying the female seat quota, \citeauthor{Chen2010} argues every one percentage point increase of female seats on the lawmaking body, the ratio of government expenditure on health and social welfare to GDP will shift up 0.18 and 0.67 percentage points.
Focusing on the health expenditure itself, \citeA{Mavisakalyan2014} studied the government and the cabinet and concludes that increasing the share of women in the administration may increase the government's spending on the health care system.

This paper tries to study the specific relationship between the legislature and health care.
Precisely speaking, this paper will investigate the relationship between the proportion of seats held by women lawmakers and the national health care expenditure.
The rest of the paper will be organized as follow.
Section \ref{Data and method} will introduce the data been used in this research and discuss the model been employed.
Section \ref{Empirical findings} will present the analytical result, and interpret the coefficient and relationship between variables.
The last section \ref{Conclusion} will discuss the conclusion from this study and also raise some limitations about the methods been used. 

